% arara: lualatex: {shell: 1}
\documentclass{article}
\newif\ifprompts{}
\promptstrue{}
\input{$HOME/Dropbox/Documents/TeX/math-pre.tex}
\begin{document}
\title{Quadratic Sieve In Rust}
\author{Nils Olsson}
\maketitle
\tableofcontents

\begin{itemize}
    \item Presentation between 12 and 15 minutes
    \item Report about/at least 6 pages.
\end{itemize}

\section*{Resources}
\begin{itemize}
    \item
        \url{https://www.researchgate.net/publication/266239994_Factoring_Integers_With_Large_Prime_Variations_of_the_Quadratic_Sieve}
\end{itemize}

\section{Introduction}

Factorizing integers is an age-old problem stemming from the fundamental theorem
of arithmetic: that every positive integer has a unique prime factorization.
Numerical number theorists have for centuries endeavoured to construct faster
factoring algorithms; one such algorithm developed within the last several
decades is the \emph{quadratic sieve}.
\begin{quote}
    The quadratic sieve algorithm (QS) is an integer factorization algorithm
    and, in practice, the second fastest method known (after the general number
    field sieve). It is still the fastest for integers under 100 decimal digits
    or so, and is considerably simpler than the number field sieve. It is a
    general-purpose factorization algorithm, meaning that its running time
    depends solely on the size of the integer to be factored, and not on special
    structure or properties. It was invented by Carl Pomerance in 1981 as an
    improvement to Schroeppel's linear sieve
    \footnote{\url{https://en.wikipedia.org/wiki/Quadratic\_sieve}}.
\end{quote}

\pagebreak
\section{Details}

Given $n$ a composite integer that is not a prime power.

\begin{itemize}
    \item
        Factor base $S=\{p_1,p_2,\ldots,p_t\}$
        where $p_1=-1$ and $p_j$ for $j\ge 2$ is the $(j-1)^\text{th}$ odd prime
        $p$ for which $n$ is a quadratic residue modulo $p$.
    \item
        $m=\floor{\sqrt n}$
    \item
        Collect $t+1$ pairs $(a_i,b_i)$
        via an $x$ chosen in the order $0,\pm 1,\pm 2,\ldots$
        satisfying $a_i^2=(x+m)^2\equiv b_i\pmod n$.
\end{itemize}

\section{Notes}

\subsection{Legendre symbol}

A multiplicative function with values 1, −1, 0
that is a quadratic character modulo an odd prime number $p$:
its value at a (nonzero) quadratic residue mod $p$ is 1
and at a non-quadratic residue (non-residue)
is −1. Its value at zero is 0

\subsection{Quadratic residue}

An integer $q$ is called a \emph{quadratic residue} modulo $n$
if it is congruent to a perfect square modulo $n$;
i.e., if there exists an integer $x$ such that:
\[
    x^2 \equiv q\pmod n.
\]
Otherwise, $q$ is called a \emph{quadratic non-residue} modulo $n$.
\end{document}

