% arara: lualatex: {shell: 1}
\documentclass[11pt,aspectratio=1610,xcolor=dvipsnames]{beamer}

\usetheme{metropolis}

\usepackage{graphicx}
\usepackage{emoji}
\usepackage{booktabs}
\usepackage{amsmath}
\usepackage{mathtools}
%\usepackage{array}
\usepackage{tabularx}
\usepackage{colortbl}
\usepackage{multicol}
\usepackage{paracol}
%\usepackage{enumitem}
%\usepackage{hyperref}
\hypersetup{
    colorlinks=true,
%     linkcolor=blue,
%     filecolor=magenta,
%     urlcolor=cyan,
}
\urlstyle{same}
\usepackage{fancyvrb}

\setemojifont{TwemojiMozilla}

\AtBeginDocument{\fontseries{l}\selectfont}

\DeclareMathOperator{\ZZ}{\mathbb{Z}}
\DeclareMathOperator{\calB}{\mathcal{B}}
\DeclareMathSymbol{\mminus}{\mathbin}{AMSa}{"39}

\DeclarePairedDelimiter{\abs}{\lvert}{\rvert}
\DeclarePairedDelimiter{\ceil}{\lceil}{\rceil}
\DeclarePairedDelimiter{\floor}{\lfloor}{\rfloor}

\newcolumntype{C}{>{\begin{math}}c<{\end{math}}}

\setlength{\parskip}{1em}

\newcommand{\emphcolor}{Red}
\newcommand{\Emph}[1]{{\color{\emphcolor}\emph{#1}}}

\title[Quadratic Sieve]{Quadratic Sieve Algorithm}
\subtitle{a.k.a.\@ the Second Fastest Algorithm in the West}

\author{Nils Olsson}
\date{May $5^\text{th}$, 2021}

\begin{document}
\begin{frame}
    \titlepage
\end{frame}

\begin{frame}
    The quadratic sieve (QS) algorithm\ldots
    \begin{itemize}
        \item Invented by Carl Pomerance in 1981 (improvement to Schroeppel's linear sieve)
        \item In practice is the second fastest integer factorization algorithm
            (after the general number field sieve)
        \item Still fastest for integers over 100 decimal digits
        \item \textbf{Fundamentally works like (random) square factoring}
    \end{itemize}
\end{frame}

\begin{frame}{(Random) Square Factoring}
    Fix $n$ an integer, and suppose we have integers $x$ and $y$ such that:
    \[
        x\ne \pm y\pmod n, \quad \text{but }
        x^2\equiv y^2 \pmod n.
    \]
    Then
    $n$ divides $(x-y)(x+y) = x^2-y^2$,
    but not $(x-y)$ or $(x+y)$ alone.

    {%
        \color{Blue}
        This means $\gcd(x\pm y,n)$ gives``non-trivial''
        (not $\pm 1$ or $\pm n$) factors of $n$.
    }

    (This is known as Fermat's factorization method.) \pause

    \textbf{The question: \Emph{how to we find $x$ and $y$?}}
\end{frame}

\begin{frame}{(Random) Square Factoring}
    (Attempt 0) \textbf{Pick randomly (a.k.a.\@ cross your fingers)}

    \begin{enumerate}
        \item Pick $x\in\ZZ$ (randomly).
        \item If $x^2\pmod n$ is a perfect square, \emoji{call-me-hand}
    \end{enumerate} \pause
    (e.g.) With $n=5959$, we randomly \emoji{winking-face} picked $x=80$.
    We find
    \begin{itemize}
        \item ${80}^2 = 6400 \equiv 441 = {21}^2 \pmod{5959}$
        \item $\gcd(80\pm 21,5959)=59\text{ and }101$,
            two non-trivial factors of 5959.
    \end{itemize} \pause
    (This might be slower than trial division)

    For $x=2,\ldots,\sqrt n$, check if $x$ divides $n$
    (keep dividing, collect powers, etc.)
\end{frame}

\begin{frame}{(Random) Square Factoring}
    (Attempt 1) \textbf{Dixon's method}

    Select a subset of primes
    $\calB = \{ p_1, p_2, \ldots, p_t \}$
    called your \Emph{factor base}.

    Any integer $x_i$ that can be written as $x_i = \prod_j p_j^{e_{ij}}$
    for $p_i\in\calB$ (or at least with $\max(e_{ij})\le\max(p_j)$)
    is called $p_t$-smooth. \pause

    We want to find pairs $(x_i,y_i)$ satisfying
    {\color{Orange} two conditions}:
    \begin{enumerate}
        \item $x_i^2 \equiv y_i\pmod n$ (\Emph{quadratic residue}), and
        \item $y_i = \prod_j p_j^{e_{ij}}$ ($y_i$ is $p_t$-smooth)
    \end{enumerate}
\end{frame}

\begin{frame}{(Random) Square Factoring}
    (Attempt 1) \textbf{Dixon's method} (continued)

    Let $\vec e_i=(e_{i1},e_{i2},\ldots)$
    and $\vec v_i=(e_{i1}\bmod 2,\ldots)$ be vectors of its exponents.

    {%
        \color{Blue}
        If a sum of some of these binary vectors is the zero vector,
        then from their corresponding $x_i$ and $y_i$'s we can construct an
        $x$ and a $p_t$-smooth $y$ such that
        \[
            x\not\equiv y\pmod n,\quad\text{but }
            x^2\not\equiv y^2\pmod n.
            \quad \text{\emoji{white-check-mark}}
        \]
    }
\end{frame}

\begin{frame}{Review (Quadratic Residue)}
    Fix $n\in\ZZ$ and $a\in\ZZ_n^*$.

    If there exists exists an $x\in\ZZ_n^*$ such that $x^2\equiv a\pmod n$
    \begin{itemize}
        \item $a$ is a \Emph{quadratic residue/square} modulo $n$;
            if there is no such $x$, then
        \item $a$ is a \Emph{quadratic non-residue} modulo $n$.
    \end{itemize}
    (e.g.) fix $a=2$
    \begin{multicols}{2}
        With $n=5$
        \begin{flalign*}
            2^2 = 4  \phantom{\equiv 1} &\not\equiv 2 \pmod 5 & \\
            3^2 = 9           \equiv 4  &\not\equiv 2 \pmod 5 & \\
            4^2 = 16          \equiv 1  &\not\equiv 2 \pmod 5 &
        \end{flalign*}
        2 is \Emph{not} a non-residue modulo 5.

        \columnbreak
        With $n=7$
        \begin{flalign*}
            2^2 = 4 \not\equiv 2 \pmod 7 && \\
            {\color{Blue}3^2 = 9    \equiv 2 \pmod 7} && \\
        \end{flalign*}
        2 \Emph{is} a quadratic residue modulo 7.
    \end{multicols}
\end{frame}

\begin{frame}{(Random) Square Factoring}
    (Attempt 1) \textbf{Dixon's method} (continued)

    If we construct a matrix $V$ from the $v_i$'s,
    we're looking for linearly dependent row-index subsets.

    (Note we have a column for each $p_j$ of the factor base $\calB$.)

    To guarantee linear dependence, we collect $\lvert\calB\rvert+1$
    such $(x_i,y_i)$ pairs. Using linear algebra we find those subsets.
    \pause

    {\color{Blue}%
        If for our constructed $x$ and $y$
        we have $x\not\equiv \pm y$,
        then for $n$ we have found non-trivial factors $\gcd(x\pm y,n)$.
    }
    \pause

    \Emph{But we can do better}
    %Alas \Emph{we're still picking $x's$ at random}
    %and testing the {\color{Orange}two conditions}.
\end{frame}

\begin{frame}{Square Factoring}
    (Attempt 2) \textbf{Quadratic sieve}

    Fix $n$ a composite that is \Emph{not a perfect-power},
    $m=\floor{\sqrt{n}}$, and consider the polynomial
    \begin{align*}
        f(x) &= x^2 - n \\
        \Rightarrow
        f(x+kp)
        &= x^2 + 2xkp + (kp)^2 - n \\
        &= f(x) + 2xkp + (kp)^2
        \equiv f(x)\pmod p.
    \end{align*} \pause
    We choose $x_i = (x+m)$ and check whether
    $y_i = f(x_i) = (x+m)^2 - n$ is $p_t$-smooth.
\end{frame}

\begin{frame}{Square Factoring}
    (Attempt 2) \textbf{Quadratic sieve} (continued)

    Things to note:
    \begin{itemize}
        \item $x_i^2 = (x+m)^2 \equiv b_i\pmod n$,
        \item if $p$ divides $b_i$, then $(x+m)^2\equiv n\pmod p$,
        \item thus $n$ is a modulo $p$.
    \end{itemize} \pause
    So $\calB$ only needs to contain $p_j$'s such that $n$ is quadratic residue
    modulo $p_j$.
\end{frame}

\begin{frame}{Square Factoring}
    (Attempt 2) \textbf{Quadratic sieve} (continued)

    To check if $y_i = f(x_i)$ is $p_t$-smooth
    we \emph{could} use trial division.
    %(like in previous methods)

    Instead we \Emph{sieve}\pause, by...
    \begin{enumerate}
        \item Constructing $Y = \{y_i=(x_i+m)^2-n\}$ for each $x_i$
            in a \Emph{sieving interval}
        \item Solving for roots $(r_1,r_2)$ of $n$ modulo $p_j\in\cal B$.
            If they exist
        \item For each $y_i\equiv r\pmod m$,
            dividing $y_i$ by $p_j$ (until no-longer divisible).
    \end{enumerate} \pause
    When complete, if $y_i=1$ then $y_i$ completely factorable over $\calB$.

    If we have at least $\abs{\calB}$ of these,
    then we proceed in the same way as Dixon's:

    Using linear algebra to find linearly dependent
    $y_i$ binary exponent vectors.
\end{frame}

\begin{frame}{Implementation}
    Notes:
    \begin{itemize}
        \item Since $n$ must be a non-perfect-power composite,
            QS factoring is merely one part of a complete factoring algorithm
            \pause
        \item \Emph{Haven't even considered how to choose $\calB$,
            or the sieving interval!}
            \pause
    \end{itemize}
\end{frame}
\begin{frame}{Implementation}
    Notes (continued):
    \begin{itemize}
        \item Many parts that in theory take we for granted:
            \begin{itemize}
                \item Factoring over the factor base
                \item Solving for quadratic roots of $n$ modulo $p$
                \item Solving for linear dependencies
                \item Primality testing
                \item Perfect-power testing
            \end{itemize}
            \pause
        \item I used (i.e.\@ implemented)
            \begin{itemize}
                \item \href
                    {https://github.com/nilsso/quadratic-sieve-rs/blob/main/python-version/qs-sieving.py\#L178}
                    {something very simple}
                \item \href
                    {https://github.com/nilsso/quadratic-sieve-rs/blob/main/python-version/sqrt_mod.py}
                    {the Tonelli-Shanks algorithm}
                \item \href
                    {https://github.com/nilsso/quadratic-sieve-rs/blob/main/python-version/qs-sieving.py\#L171}
                    {%
                        augmented matrix$\to$elementary row operations$\to$
                        echelon form to find left-nullspace
                    }
                \item Skipped this!
                \item \href
                    {https://github.com/nilsso/quadratic-sieve-rs/blob/main/python-version/quadratic-sieve.py\#L33}
                    {``Detecting Perfect Powers In Essentially Linear Time''}
            \end{itemize}
    \end{itemize}
\end{frame}
\begin{frame}{Implementation}
    (At least) Two implementations I've written in pure Python:
    \begin{itemize}
        \item \href
            {https://github.com/nilsso/quadratic-sieve-rs/blob/main/python-version/qs-sieveless.py}
            {qs-sieveless.py}
            (the ``instructional'' version)
        \item \href
            {https://github.com/nilsso/quadratic-sieve-rs/blob/main/python-version/qs-sieving.py}
            {qs-sieving.py}
    \end{itemize}
    Also planning on implementing in Rust.
\end{frame}
\end{document}
